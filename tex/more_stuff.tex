% !TEX root = ./main.tex

% Abbreviations
\usepackage{xspace}
\newcommand*{\eg}{\textit{e.g.}\@\xspace}
\newcommand*{\ie}{\textit{i.e.}\@\xspace}
\newcommand*{\etc}{\textit{etc}\@\xspace}
\newcommand*{\ade}{{\sc adenine}\@\xspace}

% MS stuff
\newcommand*{\RR}{{\em RR}\@\xspace}
\newcommand*{\SP}{{\em SP}\@\xspace}
\newcommand*{\PP}{{\em PP}\@\xspace}
\newcommand*{\PR}{{\em PR}\@\xspace}
\newcommand*{\CIS}{{\em CIS}\@\xspace}
\newcommand*{\PCOs}{{PCOs}\@\xspace}
\newcommand*{\PCO}{{\em PCO}\@\xspace}
\newcommand*{\f}{{Current Course Assignment}\@\xspace}
\newcommand*{\F}{{CCA}\@\xspace}
\newcommand*{\g}{{\PCOs Evolution Prediction}\@\xspace}
\newcommand*{\G}{{PEP}\@\xspace}
\newcommand*{\fog}{{Future Course Assignment}\@\xspace}
\newcommand*{\FOG}{{FCA}\@\xspace}


% GENE SYMBOL
\newcommand{\apc}{{\sc apc}\@\xspace}
\newcommand{\kras}{{\sc kras}\@\xspace}
\newcommand{\ctnnb}{{\sc ctnnb{\footnotesize 1}}\@\xspace}
\newcommand{\tp}{{\sc tp{\footnotesize 53}}\@\xspace}
\newcommand{\msh}{{\sc msh{\footnotesize 2}}\@\xspace}
\newcommand{\mlh}{{\sc mlh{\footnotesize 1}}\@\xspace}
\newcommand{\pms}{{\sc pms{\footnotesize 2}}\@\xspace}
\newcommand{\pten}{{\sc pten}\@\xspace}
\newcommand{\smad}{{\sc smad{\footnotesize 4}}\@\xspace}
\newcommand{\stk}{{\sc stk{\footnotesize 11}}\@\xspace}
\newcommand{\gsk}{{\sc gsk{\footnotesize 3}b}\@\xspace}
\newcommand{\axin}{{\sc axin{\footnotesize 2}}\@\xspace}

% Math stuff
\usepackage{amsmath,amsfonts,dsfont,mathrsfs,mathtools,amssymb}
\usepackage{bm}
\newcommand{\dataset}{{\cal D}}
\newcommand{\fracpartial}[2]{\frac{\partial #1}{\partial  #2}}
\newcommand{\argmin}[1]{\underset{#1}{\operatorname{arg}\,\operatorname{min}}\;}
\newcommand{\argmax}[1]{\underset{#1}{\operatorname{arg}\,\operatorname{max}}\;}
\newcommand{\norm}[1]{\left\lVert#1\right\rVert}

% Notes
%\usepackage{xcolor}
\usepackage[usenames,dvipsnames,svgnames,table]{xcolor}
\newcommand\todo[1]{\textcolor{red}{{\bf \{#1\}}}} %TODO
\newcommand{\me}[0]{Samuele Fiorini}
%\newcommand{\longtitle}[0]{Understanding time-evolving \\ biomedical data with machine learning}
%\newcommand{\longtitle}[0]{From biological questions to ...\\ }
\newcommand{\longtitle}[0]{Challenges in biomedical data science:\\data-driven solutions to clinical questions}
\newcommand{\runningtitle}[0]{Machine Learning 4 healthcare}

% Code snippets quotations and boxes
\usepackage{listings}
\usepackage{csquotes}
\usepackage{algorithm, algpseudocode}
\usepackage{tcolorbox}

\newtheorem{theorem}{Theorem}

% Images root
\graphicspath{{../images/}}

% Additional handy packages
\usepackage[T1]{fontenc}
\usepackage[english]{babel}
\usepackage{booktabs}
\usepackage[inline]{enumitem}
\usepackage{caption}
% \usepackage{subcaption}
\usepackage{longtable}
\usepackage{graphicx}
\usepackage{subfig}
% \usepackage{subfloat}
\usepackage{tikz}
\usetikzlibrary{automata,arrows,positioning,calc}


% %% Edit the Part appearance
\usepackage{xpatch}
\makeatletter
\xpatchcmd{\@part}{\huge\bfseries \partname\nobreakspace\thepart}{}{}{}
\makeatother


% biblio
\usepackage{natbib}
\renewcommand\cite{\citep}
\setcitestyle{square}

% ---------- Corradi stuff ------------------ %
\usepackage{titlesec}
\usepackage{imakeidx}
\usepackage{microtype}
\usepackage{currfile}

% For really empty page before chapter
\let\origdoublepage\cleardoublepage
\newcommand{\clearemptydoublepage}{%
  \clearpage
  {\pagestyle{empty}\origdoublepage}%
}
\let\cleardoublepage\clearemptydoublepage

\renewcommand{\chaptername}{}

\titleformat{\chapter}
  {\normalfont\LARGE\bfseries}{\thechapter}{1em}{}
\titlespacing*{\chapter}{0pt}{3.5ex plus 1ex minus .2ex}{2.3ex plus .2ex}

\usepackage{ifpdf}

\usepackage[pdftex,pagebackref,hidelinks]{hyperref}

% hyperref backref
\renewcommand*{\backref}[1]{}
\renewcommand*{\backrefalt}[4]{[{\footnotesize%
\ifcase #1 Not cited.%
\or Cited on page~#2.%
\else Cited on pages #2.%
\fi%
}]}
% ---------- end Corradi stuff ------------------ %


% url appeareance
% \usepackage{url}
\usepackage{breakurl}
\DeclareUrlCommand\ULurl{%
  \renewcommand\UrlFont{\small\ttfamily\color{blue}}%
  \renewcommand\UrlLeft{\uline\bgroup}%
  \renewcommand\UrlRight{\egroup}}

\renewcommand\url[1]{\href{#1}{\ULurl{#1}}}

% ACRONYMS
\usepackage{acro}

% probably a good idea for the nomenclature entries:
\acsetup{first-style=short}

% class `abbrev': abbreviations:
\DeclareAcronym{RR}{
  short = {\em RR},
  long  = Relapsing-Remitting Multiple Sclerosis,
  class = abbrev
}

\DeclareAcronym{SP}{
	short = {\em SP},
	long  = Secondary-Progressive Multiple Sclerosis,
	class = abbrev
}

\DeclareAcronym{PR}{
	short = {\em PR},
	long  = Progressive-Relapsing Multiple Sclerosis,
	class = abbrev
}

\DeclareAcronym{PP}{
	short = {\em PP},
	long  = Primary-Progressive Multiple Sclerosis,
	class = abbrev
}

\DeclareAcronym{CIS}{
	short = {\em CIS},
	long  = Clinically Isolated Syndrome,
	class = abbrev
}

\DeclareAcronym{PCO}{
	short = {\em PCO},
	long  = Patient Centered Outcomes,
	class = abbrev
}

\DeclareAcronym{MRI}{
	short = MRI,
	long  = Magnetic Resonance Imaging,
	class = abbrev
}

\DeclareAcronym{PET}{
	short = PET,
	long  = Positron Emission Tomography,
	class = abbrev
}

\DeclareAcronym{SPECT}{
	short = SPECT,
	long  = Single-Photon Emission Computed Tomography,
	class = abbrev
}

\DeclareAcronym{NGS}{
	short = NGS,
	long  = Next-Generation Sequencing,
	class = abbrev
}

\DeclareAcronym{EDA}{
	short = EDA,
	long  = Exploratory-Data Analysis,
	class = abbrev
}

\DeclareAcronym{GUI}{
	short = GUI,
	long  = Graphical User Interface,
	class = abbrev
}

\DeclareAcronym{API}{
	short = API,
	long  = Application Program Interface,
	class = abbrev
}

\DeclareAcronym{CT}{
	short = CT,
	long  = Computerized Tomography,
	class = abbrev
}

\DeclareAcronym{DNA}{
	short = DNA,
	long  = Deoxyribonucleic Acid,
	class = abbrev
}

\DeclareAcronym{ERM}{
	short = ERM,
	long  = Empirical Risk Minimization,
	class = abbrev
}

\DeclareAcronym{MLE}{
	short = MLE,
	long  = Maximum Likelihood Estimation,
	class = abbrev
}

\DeclareAcronym{MAP}{
	short = MAP,
	long  = Maximum A Posteriori,
	class = abbrev
}

\DeclareAcronym{LOTUS}{
	short = LOTUS,
	long  = Law Of The Unconscious Statistician,
	class = abbrev
}

\DeclareAcronym{iid}{
	short = \textit{i.i.d.},
	long  = Independent and Identically Distributed,
	class = abbrev
}

\DeclareAcronym{OLS}{
	short = OLS,
	long  = Ordinary Least Squares,
	class = abbrev
}

\DeclareAcronym{MAE}{
	short = MAE,
	long  = Mean Absolute Error,
	class = abbrev
}

\DeclareAcronym{OVO}{
	short = OVO,
	long  = \textit{One-vs-One},
	class = abbrev
}

\DeclareAcronym{OVA}{
	short = OVA,
	long  = \textit{One-vs-All},
	class = abbrev
}

\DeclareAcronym{FISTA}{
	short = FISTA,
	long  = Fast Iterative Shrinkage-Thresholding Algorithm,
	class = abbrev
}

\DeclareAcronym{SVM}{
	short = SVM,
	long  = Support Vector Machine,
	class = abbrev
}

\DeclareAcronym{SVR}{
	short = SVR,
	long  = Support Vector Regression,
	class = abbrev
}

\DeclareAcronym{SVC}{
	short = SVC,
	long  = Support Vector Classification,
	class = abbrev
}

\DeclareAcronym{RKHS}{
	short = RKHS,
	long  = Reproducing Kernel Hilbert Space,
	class = abbrev
}

\DeclareAcronym{RBF}{
	short = RBF,
	long  = Radial Basis Function,
	class = abbrev
}



%\DeclareAcronym{<ID>}{
%	short = <short> ,
%	long  = <long> ,
%	class = <class>
%}
