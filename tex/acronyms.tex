\usepackage{acro}

% probably a good idea for the nomenclature entries:
\acsetup{first-style=short}

% class `abbrev': abbreviations:
\DeclareAcronym{RR}{
  short = {\em RR},
  long  = Relapsing-Remitting Multiple Sclerosis,
  class = abbrev
}

\DeclareAcronym{SP}{
	short = {\em SP},
	long  = Secondary-Progressive Multiple Sclerosis,
	class = abbrev
}

\DeclareAcronym{PR}{
	short = {\em PR},
	long  = Progressive-Relapsing Multiple Sclerosis,
	class = abbrev
}

\DeclareAcronym{PP}{
	short = {\em PP},
	long  = Primary-Progressive Multiple Sclerosis,
	class = abbrev
}

\DeclareAcronym{CIS}{
	short = {\em CIS},
	long  = Clinically Isolated Syndrome,
	class = abbrev
}

\DeclareAcronym{PCO}{
	short = {\em PCO},
	long  = Patient Centered Outcomes,
	class = abbrev
}

\DeclareAcronym{MRI}{
	short = MRI,
	long  = Magnetic Resonance Imaging,
	class = abbrev
}

\DeclareAcronym{PET}{
	short = PET,
	long  = Positron Emission Tomography,
	class = abbrev
}

\DeclareAcronym{SPECT}{
	short = SPECT,
	long  = Single-Photon Emission Computed Tomography,
	class = abbrev
}

\DeclareAcronym{NGS}{
	short = NGS,
	long  = Next-Generation Sequencing,
	class = abbrev
}

\DeclareAcronym{EDA}{
	short = EDA,
	long  = Exploratory-Data Analysis,
	class = abbrev
}

\DeclareAcronym{GUI}{
	short = GUI,
	long  = Graphical User Interface,
	class = abbrev
}

\DeclareAcronym{API}{
	short = API,
	long  = Application Program Interface,
	class = abbrev
}


%\DeclareAcronym{<ID>}{
%	short = <short> ,
%	long  = <long> ,
%	class = <class>
%}