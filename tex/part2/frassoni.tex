% !TEX root = ../main.tex

\chapter{Tick tock: metabolic changes define a new aging clock} \label{chap:frassoni}
% Frassoni sani + CCS

%Per rispondere alla richiesta di Samuele:
%- MDA (malondialdeide) è un marcatore della perossidazione lipidica, quindi dello stress ossidativo
%- Consumo d'ossigeno e ATP sintesi con P/M: valutazione del consumo di ossigeno e sintesi di ATP in presenza di piruvato e malato che stimolano la via mitocondriale composta da Complesso I, III e IV (valuta la funzionalità del mitocondrio)
%- Consumo d'ossigeno e ATP sintesi con succinato: valutazione del consumo di ossigeno e sintesi di ATP in presenza di succinato che stimola la via mitocondriale composta da Complesso II, III e IV (valuta la funzionalità del mitocondrio)
%- ATP: concentrazione intracellulare di ATP
%- AMP: concentrazione intracellulare di AMP
%- ATP/AMP: rapporto tra le due precedenti  misure che indica lo stato energetico cellulare ( più elevato più la cellula possiede energia)
%- LDH: attività della lattico deidrogenasi, ultimo enzima del pathway della glicolisi anaerobia (antagonista della fosforilazione ossidativa che noi valutiamo tramite i consumo d'ossigeno e la sintesi di ATP).

\begin{displayquote}
	\textit{In this chapter, we study the changes of energy metabolism during the physiological aging. To this aim we measure a set of molecular biomarkers from peripheral blood mononuclear cells obtained from healthy volunteers with age between $8$ and $106$ years. With such biomarkers it is possible to quantify oxidative phosphorylation efficiency, \atpamp ratio, lactate dehydrogenase activity and level of malondialdehyde. After a thorough preliminary data exploration, we develop a regression model that, starting from such measures, is capable of predicting the age of an individual.}
\end{displayquote}

\section{Introduction: aging and metabolism} \label{sec:frassoni_intro}

In this chapter we present the first biomedical data science challenge of the thesis.
This consists in devising a ML model capable of predicting the age of an individual starting from a set of molecular biomarkers collected from $118$ volunteers\footnote{ This was referred to as \textit{the aging problem} throughout Chapter~\ref{chap:state-of-the-art}.}.

This chapter describes the extensive EDA and the thorough model selection procedures which lead to the development of the final predictive model.
Before diving into the details of the experimental setup, let's see some preliminary biological notions of how aging influences our metabolism.

Aging is a multifactorial process characterized by a progressive decline of physiological functions~\cite{campisi2013aging} which leads to an increment of vulnerability and the relative risk of disease and death ~\cite{bratic2010mitochondrial}.

Aging represents the primary risk factor for several chronic pathologies, such as cancer, cardiovascular disorders, diabetes and neurodegeneration~\cite{lopez2013hallmarks}. Different molecular pathways seem involved in the aging process, including deregulated autophagy, mitochondrial dysfunction, telomere shortening, oxidative stress, systemic inflammation and metabolism dysfunction~\cite{lopez2013hallmarks, riera2016signaling}.

For several years, aging has been considered the result of damages accumulation due to an excessive production of reactive oxygen species.
A recent paper,  \cite{thompson2017epigenetic} proposed an involvement of epigenetic modifications. This led to the development of an \textit{aging clock} based on the degree of DNA methylation, which increases with age~\cite{horvath2013dna}. 

The \textit{Mitochondrial Theory of Aging}~\cite{harman1972biologic, sastre2000mitochondrial} derives from the  concept that mitochondria are the main source of oxidative stress~\cite{cadenas2000mitochondrial, turrens2003mitochondrial, dai2014mitochondrial} and the fact that mitochondrial DNA displays a
great rate of mutation together with a less efficient repair machinery with respect to nuclear DNA~\cite{short2005decline}. After some mitochondrial DNA mutation threshold, irreversible oxidative damages propagate throughout the genome. This phenomenon leads to dysfunction of mitochondrial metabolism~\cite{genova2004mitochondrial} accelerating the oxidative stress production~\cite{wallace2010mitochondrial}.

As shown in~\cite{mckerrell2015leukemia}, mononuclear cells isolated from peripheral blood, are an excellent model to evaluate the metabolic status of an entire organism. In fact, the molecular alterations identified in peripheral blood cells of aged normal subjects are known to be statistically correlated with degenerative diseases~\cite{jaiswal2014age}.


\section{Data collection} \label{sec:frassoni_data_collection}

The study presented in this chapter is performed on mononuclear cells isolated from peripheral blood obtained from a population of $118$ volunteers\footnote{ All participants provided their written informed consent to participate in this study, which was approved by the Ethics Committee of the IRCCS Istituto G. Gaslini, Genoa, IT.} with age between $8$ and $106$ years.
In order to preserve the collected blood samples, the vacutainer tubes were transferred into the laboratory and analyzed within $24$ hours from collection.
All chemicals were purchased from Sigma Aldrich (St. Louis, MO, USA) and Ultrapure water (Milli-Q; Millipore, Billerica, MA, USA) was used throughout. All other reagents were of analytical grade.
Data collection and further analysis were managed by a team of specialized biologists at the IRCCS Istituto G. Gaslini, Genoa, IT. The following quantities were measured on each blood sample.

\begin{itemize}
	\item[] \textbf{ATP} This complex molecule is the main responsible for storing and exchanging energy in cells and it is often referred to as the \textit{energy currency} of the cell. From a chemical point of view, \ATP is made of an adenine base attached to a ribose sugar, which, in turn, is attached to three phosphate groups.
	\ATP is heavily involved in the cellular aerobic respiration pathway. High levels of \ATP correspond to high energetic state.
	\ATP intracellular concentration, measured in $\text{mM}/\text{ml}$, is an important molecular biomarker to evaluate the energetic state of a cell.
	
	\item[] \textbf{AMP} This molecule is one of the main derivatives of \ATP. In fact, \AMP ca be obtained when two phosphate groups are removed from \ATP, releasing energy that can be transferred to other molecules to trigger further cell reactions.
	So, when a cell is in good health, \ie high energetic level, \AMP is low and \ATP is high.
	\AMP intracellular concentration is measured in $\text{mM}/\text{ml}$ and it is considered as an important molecular biomarker for the cellular energetic state. \ATP and \AMP quantification was based on the enzyme coupling method presented in~\cite{ravera2013tricarboxylic}.
%	as a side product of the ATP synthesis process or 
	
	\item[] \textbf{ATP/AMP ratio} Measuring \ATP and \AMP cellular concentration may not be enough to predict the age of an individual by assessing the energetic state of the peripheral blood cells. A more representative and interesting quantity can be their ratio, so \atpamp ratio was calculated and added to the feature set.
	
	\item[] \textbf{Oxygen consumption} Aerobic cellular respiration requires oxygen to produce \ATP. Therefore, the cellular oximetric level is an important molecular biomarker for the metabolic assessment. Oxygen consumption was measured with an amperometric electrode in a closed chamber, magnetically stirred, at $37$°C. The oxygen consumption measure, expressed in $\text{nmol~O}_2/(\text{min}\cdot\text{mg})$, is repeated in two versions, adding two different substrates, \ie:
	\begin{enumerate*}[label=(\roman*)]
		\item a combination of $5~\text{mM}$ pyruvate with $2.5~\text{mM}$ malate or
		\item $20~\text{mM}$ succinate.
	\end{enumerate*}
	In the first oximetric measure, which from now on will be referred to as \copyrmal, the substrate stimulates the pathway composed by Complexes I, III and IV. On the other hand, in the second oximetric measure, which we call \cosucc, the substrate activates the pathway composed by Complexes II, III and IV, as described in~\cite{cappelli2017defects}.
	
	\item[] \textbf{ATP synthesis} We already covered the importance of ATP to evaluate the metabolic state of cells. So, we also measured ATP synthesis, expressed in $\text{nmol~ATP}/(\text{min}\cdot\text{mg})$,  by the highly sensitive luciferin/luciferase method. The same two substrates used for the oximetric evaluation were adopted. In the remainder of the chapter we refer to such measures as \atppyrmal and \atpsucc, accordingly.
	
	\item[] \textbf{P/O ratio} In order to assess the efficiency of oxidative phosphorylation, we also evaluated the ratio between \ATP synthesis and oxygen consumption under both the substrates. We call the two obtained features as \popyrmal and \posucc, respectively.
	
	\item[] \textbf{Glycolytic flux} In order to assay the glycolytic flux, we measured the activity of the Lactate Dehydrogenase (\ac{LDH}), expressed in $\text{U}/\text{mg}$. This enzyme is important to evaluate the other metabolic pathway not mentioned so far, \ie the anaerobic respiration.
	
	\item[] \textbf{Lipid peroxidation} The uncoupled oxidative phosphorylation metabolism is often associated with an increment in the oxidative stress production~\cite{dai2014mitochondrial}, which induces damages on proteins, nucleic acid and membrane. Therefore, we evaluated the level of malondialdehyde (\ac{MDA}) as a marker of lipid peroxidation. The measure of \mda, expressed in $\mu\text{M}/\text{mg}$, follows the protocol in~\cite{ravera2015oxidative}.
\end{itemize}

The categorical feature \gender is encoded as $[0,1]$ for \textit{male} ad \textit{female}, respectively.
Each example of this dataset is then described by the $12$-dimensional feature set summarized in Table~\ref{tab:aging_features}.

\begin{table}[]
	\centering
	\caption{The $12$-dimensional feature set.}
	\label{tab:aging_features}
	\begin{tabular}{@{}ll@{}}
		\toprule
		\textbf{Measure}                                    & \textbf{Feature name}\\ \midrule
		Gender of the individual                      & \gender          \\
		ATP intracellular concentration            & \atp         \\
		AMP intracellular concentration            & \amp         \\
		ATP/AMP ratio                              & \atpamp      \\
		Oxygen consumption under pyruvate + malate & \copyrmal    \\
		Oxygen consumption under succinate         & \cosucc      \\
		ATP synthesis under pyruvate + malate      & \atppyrmal   \\
		ATP synthesis under succinate              & \atpsucc     \\
		P/O ratio under pyruvate + malate          & \popyrmal    \\
		P/O ratio under succinate                  & \posucc      \\
		Glycolytic flux                            & \ldh         \\
		Lipid peroxidation                         & \mda         \\ \bottomrule
	\end{tabular}
\end{table}


\section{Exploratory data analysis} \label{sec:frassoni_EDA}
% BOXPLOT - PCA - HEATMAP etc
In this section we investigate on the relationship between the collected molecular biomarkers and the age of $118$ healthy individuals which volunteered to participate to this study.
Unfortunately, $7$ subjects presented missing values (in the features \ldh and \gender). These subjects were excluded from EDA.


The data collection process entirely ran on voluntary basis. So, it is interesting to observe the resulting age distribution. As we can see from the histogram in Figure~\ref{fig:frassoni_agehist}, the decades of age are not equally represented. Ideally, we would have collected samples uniformly distributed with respect to their age, but this was unfortunately not possible. Therefore, in this data science challenge we shall adopt robust resampling schemes in order to ameliorate possible biases induced by this phenomenon.
%Therefore, we shall provide appropriate countermeasures to avoid bias in the following supervised regression step \todo{false so far}.

\begin{figure}[]
	\centering
	\includegraphics[width=0.8\textwidth]{part2/aging_agehist.png}
	\caption{Age distribution of the $118$ individuals involved in the study.} \label{fig:frassoni_agehist}
\end{figure}

Next, we aim at investigating on how the distribution of the molecular biomarkers is influenced by the age of individuals. To this aim we perform a preliminary univariate analysis. We group the measures per decade and we represent their distribution with boxplots, see Figure~\ref{fig:frassoni_boxplot}.
As we can see, most of the biomarkers are clearly influenced by the age.
Let's start the visual inspection from the variables related to the mitochondrial aerobic metabolism, \ie Figure~\ref{fig:frassoni_boxplot_copyrmal} to \ref{fig:frassoni_boxplot_atpamp}.

In particular, focusing our attention on the \atpamp ratio (Figure~\ref{fig:frassoni_boxplot_atpamp}), which is known to be an energy status monitor of the cells, we can see that the values decrease progressively with the decades, with a drastic drop between $40$ and $50$ years. Moreover, from an observation of \atp and \amp intracellular concentration (Figure~\ref{fig:frassoni_boxplot_atp} and Figure~\ref{fig:frassoni_boxplot_amp}, respectively) we can sense how the decrease of the \atpamp ratio is mainly due to the growth of \amp in the aging process.
Similar considerations can be made for the efficiency of oxidative phosphorylation, evaluated by \popyrmal and \posucc. In particular, \popyrmal oscillates around its reference level of $2.5~\text{nmol~O}_2/(\text{min}\cdot\text{mg})$~\cite{hinkle2005p} in subjects having from $0$ to $30$ years starting to decrease afterwards. Moreover, \popyrmal, similarly to \atpamp, is fairly stable at its lowest level for elderly (age $\geq 70$).

Let's now focus on the activity of \LDH (Figure~\ref{fig:frassoni_boxplot_ldh}). As we can see, this metabolic biomarker almost monotonically increases with the aging process.
This is mainly due to the fact that \LDH is involved in the glycolysis metabolism, which is a metabolic pathway chosen by the cell to compensate its altered aerobic metabolism.
Finally, let's focus on the distributions of \mda (Figure~\ref{fig:frassoni_boxplot_mda}). As expected, \mda has an opposite trend with respect to \atpamp and \popyrmal. In fact, it increases from $21$ to $80$ years. This can be due to the increased oxidative stress production induced by uncoupled oxidative phosphorylation~\cite{dai2014mitochondrial}.
\mda is more stable for elderly, mainly because of their physiological metabolic slowdown.

\begin{figure}[]
	\centering
	\subfloat[]{%
		\includegraphics[width=0.5\textwidth]{part2/aging_boxplot_CO-PyrMal.png}
		\label{fig:frassoni_boxplot_copyrmal}%
	}%
	\subfloat[]{%
		\includegraphics[width=0.5\textwidth]{part2/aging_boxplot_CO-Succinate.png}
		\label{fig:frassoni_boxplot_cosucc}%
	}%
	\hfill%
	\subfloat[]{%
		\includegraphics[width=0.5\textwidth]{part2/aging_boxplot_ATP-PyrMal.png}
		\label{fig:frassoni_boxplot_atppyrmal}%
	}%
%	\hfill% %%%% ROW 2
	\subfloat[]{%
	   \includegraphics[width=0.5\textwidth]{part2/aging_boxplot_ATP-Succinate.png}
       \label{fig:frassoni_boxplot_atpsucc}%
	}%
   \hfill
	\subfloat[]{%
		\includegraphics[width=0.5\textwidth]{part2/aging_boxplot_PO-PyrMal.png}
		\label{fig:frassoni_boxplot_popyrmal}%
	}%
%    \hfill%
	\subfloat[]{%
		\includegraphics[width=0.5\textwidth]{part2/aging_boxplot_PO-Succinate.png}
		\label{fig:frassoni_boxplot_posucc}%
	}%
%   \hfill%%%% ROW 3
\end{figure}

\begin{figure}[]
	\ContinuedFloat
    \centering
	\subfloat[]{%
		\includegraphics[width=0.5\textwidth]{part2/aging_boxplot_ATP.png}
		\label{fig:frassoni_boxplot_atp}%
	}%
	\subfloat[]{%
		\includegraphics[width=0.5\textwidth]{part2/aging_boxplot_AMP.png}
		\label{fig:frassoni_boxplot_amp}%
	}%
		\hfill%
	\subfloat[]{%
		\includegraphics[width=0.5\textwidth]{part2/aging_boxplot_ATPAMP.png}
		\label{fig:frassoni_boxplot_atpamp}%
	}%
%	\hfill%%%% ROW 4
	\subfloat[]{%
		\includegraphics[width=0.5\textwidth]{part2/aging_boxplot_MDA.png}
		\label{fig:frassoni_boxplot_mda}%
	}%
		\hfill%
	\subfloat[]{%
		\includegraphics[width=0.5\textwidth]{part2/aging_boxplot_LDH.png}
		\label{fig:frassoni_boxplot_ldh}%
	}%
	\caption{Distribution of the collected molecular biomarker values grouped per decade: \copyrmal panel (a), \cosucc panel (b), \atppyrmal panel (c), \atpsucc panel (d), \popyrmal panel (e), \posucc panel (f), \atp panel (g), \amp panel (h), \atpamp panel (i), \mda panel (j) and \ldh panel (k).} \label{fig:frassoni_boxplot}
\end{figure}

So far, we have investigated on the relationship between the collected measures and the age of the individuals. Let's now focus on the relationship between the variables themselves. In order to investigate on possible collinearities in the data, we can evaluate for each pair of variables the Pearson correlation coefficient
$$
	\rho(a, b) = \frac{\sum_{i=1}^n (a_i - \bar a)(b_i - \bar b)}{\sqrt{\sum_{i=1}^n(a_i-\bar a)^2 \sum_{i=1}^n(b_i-\bar b)^2}}
$$
where $\bar c = \frac{1}{n}\sum_{i=1} c_i$ is the empirical mean any variable $c$.
To ease the visualization, we restrict this analysis to a subset of $5$ molecular biomarkers, namely: \popyrmal, \posucc, \atpamp, \ldh, \mda.
This results in a symmetric $5 \times 5$ correlation matrix.
%This results in a symmetric $11 \times 11$ correlation matrix (excluding the categorical feature \gender).
%In order to evaluate a specific age fingerprint,
We split the data in $5$ groups, one for each two-decades, and we represent the collinearity in each group with a symmetric heatmap in which dark red cells are associated with strong positive correlation, white cells represent no correlation and dark blue cells correspond to strong negative correlation, see Figure~\ref{fig:heatmaps}.
Thanks to this visualization, we can see that \atpamp and \popyrmal have positive correlation until the $6^{\text{th}}$ decade, while showing a negative correlation for elderly.
This suggests that, up to approximately $60$ years, most of the cellular energy is produced by the mitochondria, while in older subjects this contribution decreases. The same observation can be made for \atpamp and \posucc, although in this case it is less evident. On the other hand, the correlation between \atpamp and \mda or \ldh activity is negative in young subjects, while flipping it sign after approximately $60$ years.
This is in line with the cellular need to increase the anaerobic metabolism that compensates the inefficiency of aerobic metabolism and the increment of oxidative stress which usually occurs for elderly.

\begin{figure}[]
	\centering
	\subfloat[]{%
		\includegraphics[draft=false, width=0.5\textwidth]{part2/aging_heatmap_years range_0.png}
		\label{fig:heatmaps1}%
	}%
	\subfloat[]{%
		\includegraphics[draft=false, width=0.5\textwidth]{part2/aging_heatmap_years range_1.png}
		\label{fig:heatmaps2}%
	}%
		\hfill%
	\subfloat[]{%
		\includegraphics[draft=false, width=0.5\textwidth]{part2/aging_heatmap_years range_2.png}
		\label{fig:heatmaps3}%
	}%
%	\hfill% %%%% 
	\subfloat[]{%
		\includegraphics[draft=false, width=0.5\textwidth]{part2/aging_heatmap_years range_3.png}
		\label{fig:heatmaps4}%
	}%
   \hfill
	\subfloat[]{%
		\includegraphics[draft=false, width=0.5\textwidth]{part2/aging_heatmap_years range_4.png}
		\label{fig:heatmaps5}%
	}%
	\caption{The $5 \times 5$ symmetric heatmaps representing the Pearson correlation coefficient of the collected variables in the $5$ age groups: $[1-20]$ (a), $[21-40]$ (b), $[41-60]$ (c), $[61-80]$ (d), $>81$ (e).} \label{fig:heatmaps}
\end{figure}


Let's now try to visualize the collected data in a scatter plot. In order to do that, we shall first reduce the dimensionality of the problem, as described in Section~\ref{sec:dimred}.
In this EDA, most of the variables showed strong inner linear correlation and correlation with the age. We then reduce the dimensionality of the problem following a two step pipeline:
\begin{enumerate*}[label=(\roman*)]
	\item data standardization followed by
	\item linear PCA.
\end{enumerate*}
Our hope is to recognize some quasi-linear temporal structure. The resulting scatter plot is presented in Figure~\ref{fig:frassoni_PCA}.
Each point in the scatter plot represents a subject that is color coded according to the age.
Let's read the image from left to right.
As we can see, it looks like the subjects are partially grouped according to their age. In the top left corner of Figure~\ref{fig:frassoni_PCA} subjects with approximately $20$ years (or less) are clustered together. At the center bottom of the plot we can recognize individuals around their thirties. Then, following an approximately linear law from center bottom to top right we can see that the age increases until reaching the elderly, color coded in dark red.
The insightful data visualization in Figure~\ref{fig:frassoni_PCA}, lets us sense that it is possible to devise a supervised strategy to predict the age of a given individual from the collected molecular biomarkers.

\begin{figure}[]
	\centering
	\includegraphics[width=0.8\textwidth]{part2/aging_PCA.png}
	\caption{Scatter plot obtained after projecting the data in a $3D$ space via linear PCA. The color-coding represents the age of the individuals.} \label{fig:frassoni_PCA}
\end{figure}


\section{Metabolic age prediction} \label{sec:frassoni_regression}
% pipeline + assessment + model challenge
This section presents the development of the regression model for the age prediction task presented in this chapter.

To this aim, we imputed the small fraction of missing values following the $k$-nearest neighbors (with $k=3$) proposed in~\cite{troyanskaya2001missing}. So, the development of the predictive model can take advantage of the total number of individuals ($n=118$).

The fist question we pose is whether the collected dataset is large \textit{enough} for our purpose, or we should keep enrolling new volunteers. Providing a definitive answer for this question is, in general, unfeasible. Nevertheless, we can investigate in this direction by evaluating the so-called \textit{learning curves}~\cite{murphy2012machine}. Such graphical representation is obtained by iteratively fitting a given regression model on increasingly large chunks of the entire dataset.
Horizontal axis corresponds to the number of training samples, while vertical axis represents mean values of some cross-validated regression metric (such as MAE).
The shape of the obtained learning curve tells a lot about the prediction problem at hand. For instance, if the cross-validation error keeps decreasing we may sense that there may be \textit{more} to learn about the input/output relationship and that the regression performance would benefit from a larger dataset. Conversely, if the cross-validation error initially decreases eventually reaching a plateau it may be that the number of samples is large enough and collecting more data would only marginally improve the prediction performance. This may happen mainly because:
\begin{enumerate*}[label=(\roman*)]
	\item the available feature set only partially explains the input/output relationship and more/better variables should be observed to improve the regression performance,
	\item the selected model is incapable of capturing some of the input/output relationship already hidden in the data or
	\item noise affecting the data.
\end{enumerate*}
Learning curves plot gives also information on the amount of bias and variance affecting a given predictive model.
Four our purpose, we rely on the use of standard Ridge regression model (see Section~\ref{sec:ridge_regression}) and 100-splits Monte Carlo cross-validated MAE to evaluate each point of our learning curves (see Section~\ref{sec:frassoni_results}).

In EDA we realized that most of the input variables show a quasi-linear relationship with the age. Moreover, we also realized that many molecular biomarkers present a strong pairwise linear correlation. Therefore it seems sensible to investigate whether a polynomial feature expansion of degree $D$ could be beneficial for the age prediction.
To this aim, we devised the following experimental design.

A cross-validated estimate of the empirical distribution of four regression scores: $\text{R}^2$, MAE, MSE and EV (defined in Section~\ref{sec:performance_metrics}) is evaluated on $500$ Monte Carlo random splits.
%\todo{should have been stratified -> correggere}.
This strategy consists in iteratively extracting $n_{\text{train}} = 0.75 \cdot n$ random samples multiple times\footnote{ The fraction of data to use for training is chosen accordingly to the learning curve in Figure~\ref{fig:frassoni_learning_curves}.}. On each obtained training set a supervised learning model is fitted.
%In our experiments, we ran a total of $500$ Monte Carlo resampling.
%\todo{As not each age decade is equally represented, see Figure~\ref{fig:frassoni_agehist}, we opted for a cross-validation strategy that is stratified with respect to the decades. So, in each random extraction, the same proportion of samples}
Once the model is fitted, the five regression scores are evaluated on the remaining $n_{\text{test}} = n - n_{\text{train}}$ samples.
The adopted supervised model consists in a pipeline having two steps:
\begin{enumerate*}[label=(\roman*)]
	\item data standardization and
	\item regression model fitting.
\end{enumerate*}
The first step simply consists in subtracting the mean from each feature and dividing them by their standard deviation. Nesting this preprocessing step inside the Monte Carlo cross-validation scheme improves the empirical estimate of the regression scores. In fact, for each cross-validation iteration, the mean and the standard deviation are estimated from the training set only. This allows to have, each time, a genuine score, estimated only on data points that were never seen before by the current supervised regression model.
Moreover, for the second step of the pipeline we adopted the following regression models:
\begin{enumerate*}[label=(\alph*)]
	\item Ridge,
	\item Lasso,
	\item Elastic-Net,
	\item Linear SVM,
	\item RBF Kernel SVM,
	\item RBF Kernel Ridge,
	\item Random Forests,
	\item Gradient Boosting and
	\item Multilayer Perceptron.
\end{enumerate*}
The free parameters of each model, including the degree of the polynomial expansion $D$, are optimized via grid-search $5$-fold cross-validation.

This Monte Carlo cross-validated procedure is evaluated two times: with and without a preliminary polynomial feature expansion. We expect linear models to benefit more from the polynomial expansion than the nonlinear ones.
Then, for each model and for each metric, we evaluated the p-value obtained from the one-tailed two-sample Kolmogorov–Smirnov test~\cite{everitt2002cambridge}. This let us understand in which cases the scores obtained after a polynomial expansion are significantly better than their counterpart, obtained only with  linear features.

The final goal of this section is to find the best age prediction model. Therefore, it is important to understand which are the most predictive variables.
So, as a side result of the previous analysis, we ranked the features according to their selection frequency. The most important features are more likely to be selected more often. This strategy is known in literature as  \textit{selection stability framework}~\cite{meinshausen2010stability, barbieri16palladio} and, for our purposes, it let us get a preview of which features will likely appear in the final model.

Finally, the proposed age predicting model is achieved by fitting the best regression strategy on a training set obtained by randomly extracting $75\%$ of the whole dataset. This model is eventually tested, by the usual metrics, on the remaining $25\%$. We expect to achieve regression metrics in line with the previous experiments.

\section{Results} \label{sec:frassoni_results}
% tabella dei risultati
Figure~\ref{fig:frassoni_learning_curves} shows the obtained learning curves. Interestingly, the average cross-validation error (blue line), reaches a plateau around $\text{MAE} \approx 9$ years a little bit after $80$ samples. This suggests that adding more training examples would not significantly improve the predictions. This justifies the choice of dimensioning the Monte Carlo training sets as $75\%$ of the input data (see previous section).
Moreover, we can see that the gap between training and cross-validation error becomes smaller as the number of sample increases. This suggests that the linear regression model is generalizing well. We can also notice that the training variance decreases from left to right. This suggests that, as expected, increasing the number of training examples induces a stabilization effect on the learned function.
We repeated the learning curve experiment also after the degree $D=2$ polynomial feature expansion, see Figure~\ref{fig:frassoni_learning_curves_poly}. At a first glance, we can see two main differences:
\begin{enumerate*}[label=(\roman*)]
	\item at the right hand side of the plot, the cross-validation curve (blue line) is still, slowly, decreasing and it is not reaching a plateau,
	\item the gap between training and cross-validation curves is wider.
\end{enumerate*}
Therefore, we suspect that increasing the number of collected sample would significantly improve the regression performance in this case.
For consistency with the previous case, we opted to anyway randomly extract training sets of $75\%$ of the dataset after the polynomial expansion as well.
What we describe here is a real-world case study and, at the time of writing, expanding the data collection is unfortunately not possible. Nevertheless, we strongly believe that it would be beneficial for the study, therefore we mark this as future work.


\begin{figure}[]
	\centering
	\subfloat[]{%
		\includegraphics[width=0.5\textwidth]{part2/aging_learning_curve_linfeat.png}
		\label{fig:frassoni_learning_curves}%
	}%
	\subfloat[]{%
		\includegraphics[width=0.5\textwidth]{part2/aging_learning_curve_polyfeat.png}
		\label{fig:frassoni_learning_curves_poly}%
	} %
	\caption{Learning curves obtained on the aging problem by fitting a Ridge regression model on $100$ Monte Carlo random splits and evaluating the MAE on each training (orange line) and test (blue line) sets. Panel (a) shows the learning curves obtained using only the original features, whereas panel (b) shows the results achieved after a degree $2$ polynomial feature expansion.}
\end{figure}

% !TEX root = ../main.tex

% \usepackage{pifont}% http://ctan.org/pkg/pifont
% \newcommand{\cmark}{\ding{51}}%
% \newcommand{\xmark}{\ding{55}}%

\begin{table}[h!]
\begin{tabular}{llllll}
\toprule
{} &                   PF & MAE              &                                                MSE &                                             EV &                                             R2 \\
\midrule
Ridge                  &         \xmark       &  8.848$\pm$1.172 &                                 141.042$\pm$43.490 &                                0.822$\pm$0.061 &                                0.813$\pm$0.063 \\
Lasso                  &         \xmark       &  8.757$\pm$1.185 &                                 134.290$\pm$40.427 &                                0.828$\pm$0.058 &                                0.821$\pm$0.059 \\
Elastic-Net            &         \xmark       &  8.824$\pm$1.193 &                                 137.735$\pm$42.660 &                                0.824$\pm$0.060 &                                0.816$\pm$0.062 \\
Linear SVM             &         \xmark       &  8.717$\pm$1.394 &                                 149.784$\pm$55.156 &                                0.804$\pm$0.084 &                                0.797$\pm$0.085 \\
RBF Kernel SVM                &         \xmark       &  7.856$\pm$1.354 &                                 126.292$\pm$47.623 &                                0.839$\pm$0.064 &                                0.830$\pm$0.067 \\
RBF Kernel Ridge       &         \xmark       &  8.369$\pm$1.306 &                                 143.932$\pm$56.301 &                                0.816$\pm$0.074 &                                0.808$\pm$0.077 \\
Random Forests         &         \xmark       &  \textbf{7.662$\bm{\pm}$1.259} &                                 \textbf{121.155$\bm{\pm}$44.465} &                                \textbf{0.844$\bm{\pm}$0.060} &                                \textbf{0.838$\bm{\pm}$0.062} \\
Gradient Boosting      &         \xmark       &  8.255$\pm$1.314 &                                 146.311$\pm$56.863 &                                0.814$\pm$0.080 &                                0.804$\pm$0.084 \\
MLP                    &         \xmark       &  9.995$\pm$1.396 &                                 166.300$\pm$47.473 &                                0.790$\pm$0.075 &                                0.774$\pm$0.078 \\
\midrule
Ridge                  &       \cmark      &  8.210$\pm$1.274 &                                 135.935$\pm$52.264 &                                0.825$\pm$0.077 &                                0.818$\pm$0.079 \\
Lasso                  &       \cmark      &  7.868$\pm$1.214 &                                 \textbf{116.609$\pm$39.335} &                      \textbf{0.852$\pm$0.055} &                        \textbf{0.844$\pm$0.057} \\
Elastic-Net            &       \cmark      &  7.869$\pm$1.294 &                                 118.265$\pm$40.513 &                                0.847$\pm$0.056 &                                0.841$\pm$0.058 \\
Linear SVM             &       \cmark      &  \textbf{7.543$\pm$1.286} &                        120.029$\pm$46.968 &                                0.845$\pm$0.068 &                                0.838$\pm$0.070 \\
RBF Kernel SVM                &       \cmark      &  7.905$\pm$1.344 &                                 131.000$\pm$49.338 &                                0.829$\pm$0.071 &                                0.821$\pm$0.073 \\
RBF Kernel Ridge       &       \cmark      &  8.222$\pm$1.326 &                                 141.229$\pm$54.793 &                                0.817$\pm$0.079 &                                0.808$\pm$0.083 \\
Random Forests         &       \cmark      &  7.607$\pm$1.225 &                                 118.179$\pm$43.992 &                                0.849$\pm$0.058 &                                0.841$\pm$0.060 \\
Gradient Boosting      &       \cmark      &  8.190$\pm$1.455 &                                 141.996$\pm$58.447 &                                0.820$\pm$0.077 &                                0.811$\pm$0.081 \\
MLP                    &       \cmark      &  9.395$\pm$1.776 &                                183.528$\pm$107.976 &                                0.762$\pm$0.148 &                                0.750$\pm$0.156 \\
\bottomrule
\end{tabular}
\caption{The performance assessment of various ML methods on the problem described in this chapter. Values are expressed as: mean $\pm$ standard deviation. The PF flag is \cmark~for metrics obtained after a second degree polynomial features expansion, and \xmark~for linear features only. Bold digits correspond to column-wise best values.} \label{tab:frassoni_model_challenge}
\end{table}







%linear_regression      &                9.071$\pm$1.203 &                                 148.104$\pm$45.651 &                                0.811$\pm$0.065 &                                0.802$\pm$0.067
%linear_regression_poly &  15760434.933$\pm$85613510.386 &  211045874537134752.000$\pm$1604353548204884480... &  -299755011361541.438$\pm$2262628332868860.000 &  -310919237346499.562$\pm$2346898802668675.000

% !TEX root = ../main.tex

\begin{table}
\begin{tabular}{lllll}
\toprule
{} & p-value MAE & p-value MSE & p-value EV & p-value R2 \\
\midrule
Ridge              &    \cellcolor{green!25}$4.44\cdot 10^{-14}$ &    \cellcolor{green!25}$6.22\cdot 10^{-03}$ &   \cellcolor{green!25}$5.06\cdot 10^{-03}$ &   \cellcolor{green!25}$6.22\cdot 10^{-03}$ \\
Lasso              &    \cellcolor{green!25}$1.09\cdot 10^{-28}$ &    \cellcolor{green!25}$3.39\cdot 10^{-12}$ &   \cellcolor{green!25}$8.49\cdot 10^{-12}$ &   \cellcolor{green!25}$3.27\cdot 10^{-11}$ \\
Elastic-Net        &    \cellcolor{green!25}$1.76\cdot 10^{-22}$ &    \cellcolor{green!25}$9.92\cdot 10^{-10}$ &   \cellcolor{green!25}$9.92\cdot 10^{-10}$ &   \cellcolor{green!25}$4.96\cdot 10^{-09}$ \\
Linear SVM         &    \cellcolor{green!25}$6.20\cdot 10^{-26}$ &    \cellcolor{green!25}$5.07\cdot 10^{-11}$ &   \cellcolor{green!25}$4.33\cdot 10^{-10}$ &   \cellcolor{green!25}$2.84\cdot 10^{-10}$ \\
RBF SVM            &    \cellcolor{red!25}$4.65\cdot 10^{-01}$ &    \cellcolor{red!25}$9.54\cdot 10^{-02}$ &   \cellcolor{red!25}$1.96\cdot 10^{-02}$ &   \cellcolor{red!25}$8.29\cdot 10^{-02}$ \\
RBF Kernel Ridge   &    \cellcolor{red!25}$3.28\cdot 10^{-02}$ &    \cellcolor{red!25}$1.80\cdot 10^{-01}$ &   \cellcolor{red!25}$2.75\cdot 10^{-01}$ &   \cellcolor{red!25}$2.75\cdot 10^{-01}$ \\
Random Forests     &    \cellcolor{red!25}$2.75\cdot 10^{-01}$ &    \cellcolor{red!25}$3.01\cdot 10^{-01}$ &   \cellcolor{red!25}$3.28\cdot 10^{-01}$ &   \cellcolor{red!25}$3.81\cdot 10^{-01}$ \\
Gradient Boosting  &    \cellcolor{red!25}$1.60\cdot 10^{-01}$ &    \cellcolor{red!25}$7.18\cdot 10^{-02}$ &   \cellcolor{red!25}$1.25\cdot 10^{-01}$ &   \cellcolor{red!25}$7.18\cdot 10^{-02}$ \\
MLP                &    \cellcolor{green!25}$1.95\cdot 10^{-13}$ &    \cellcolor{green!25}$2.33\cdot 10^{-08}$ &   \cellcolor{green!25}$8.31\cdot 10^{-07}$ &   \cellcolor{green!25}$8.31\cdot 10^{-07}$ \\
\bottomrule
\end{tabular}
\caption{One-tailed p-value resulting from the two-sample Kolmogorov-Smirnov test used to investigate whether or not these ML methods perform better after a second degree polynomial expansion. The green cells indicate the cases that benefit from the polynomial expansion (p-value $< 0.01$).}
\end{table}






% linear\_regression &   2.25\cdot 10^{-221 &   2.25\cdot 10^{-221 &  2.25\cdot 10^{-221 &  2.25\cdot 10^{-221 \\


Table~\ref{tab:frassoni_model_challenge} shows the results of the model assessment performed as described earlier. Focusing on linear features (top half of the table), we can see that Random Forests is the top performing method (in bold). We can also notice that RF is immediately followed by SVM with RBF kernel. This suggests that some nonlinear input/output relationship is hidden in the data. So, let's see whether a simple $2^\text{nd}$-degree polynomial expansion exposes such hidden data structure.
Looking at the bottom half of the Table~\ref{tab:frassoni_model_challenge}, we can see that all the linear model benefit from the polynomial expansion. Quite surprisingly, two linear methods, namely linear SVM and the Lasso, even outperform Random Forests, according to almost every metric.

Table~\ref{tab:frassoni_pvalues} highlights the cases in which the metrics evaluated after the polynomial expansion are statistically significantly better than their counterpart evaluated with linear features only (with p-value $< 0.01$).
As expected, the polynomial expansion is beneficial for all the linear method, while it is not for almost every nonlinear method.
A separate comment can be made for MLP. In this context, given the reduced number of training samples, we opted for a shallow network with only one hidden layer. The number of hidden units in the hidden layer is considered as a free parameter of the model and it is therefore chosen via grid-search cross validation.
As we can see from Table~\ref{tab:frassoni_model_challenge}, MLP is consistently the worst performing regression method. This can partially be explained with the fact that neural networks are known to be more powerful when trained with large datasets, which is not the case. Interestingly, MLP is the only nonlinear method that statistically significantly benefits from the polynomial feature expansion.
This result looks quite surprising at a first glance, but actually one can speculate that the effect of the polynomial expansion could be compared to the effect of the addition of another hidden layer. In fact, it is known that deeper architectures, even if harder to train, usually lead to better results.

\begin{figure}[h!]
	\centering
	\includegraphics[width=0.8\textwidth]{part2/aging_topfeatures.png}
	\caption{The feature ranking obtained via stability selection on the aging problem adopting the Lasso regression model.} \label{fig:frassoni_topfeat}
\end{figure}

\begin{figure}[h!]
	\centering
	\includegraphics[width=0.8\textwidth]{part2/aging_lassopolyfeat2.png}
	\caption{The coefficients of the proposed age prediction model.} \label{fig:frassoni_topmodel}
\end{figure}

From the insights above it is clear that the proposed model uses a degree $2$ polynomial feature expansion. However, Table~\ref{tab:frassoni_model_challenge} indicates two possible linear regression models: linear SVM and the Lasso. In this context we choose to use the Lasso to accomplish the age prediction task for two main reasons:
\begin{enumerate*}[label=(\roman*)]
	\item it outperforms linear SVM for $3$ metrics out of $4$ and
	\item its sparsity enforcing penalty leads to a more compact and interpretable model.
\end{enumerate*}

Analyzing the feature selection frequency of the Lasso, Figure~\ref{fig:frassoni_topfeat}, we can see that all the top-ranked variables are features that arise from the polynomial expansion. This is a further evidence of the importance of such step.

The final Lasso model, fitted on a $75\%$ of the dataset after the $2^{\text{nd}}$ degree polynomial expansion and a feature-wise standardization leads to the coefficients of Figure~\ref{fig:frassoni_topmodel}. Such model, evaluated on the test set, achieves $\text{MAE}=6.40$ years, $\text{MSE} = 97.24$ years$^2$, $\text{R}^2 = 0.81$ explaining the $82\%$ of the variance. Which is consistent with what we expected from Table~\ref{tab:frassoni_model_challenge}.

Nevertheless, as the Lasso regression model is trained on preprocessed data, the coefficients in Figure~\ref{fig:frassoni_topmodel} cannot be used to predict the age from raw measures. To this purpose, we also report the prediction model, with its raw data-ready coefficients, as in Equation~\eqref{eq:frassoni_lasso}.

\begin{equation}\label{eq:frassoni_lasso}
\begin{aligned}
\hat{\text{age}} = {} & 1.13 \times 10^{-1} \cdot (\text{\gender} \cdot \text{\copyrmal}) +
9.63 \times 10^{-3} \cdot (\text{\cosucc} \cdot \text{\mda}) - \\
& 1.91 \times 10^{-1} \cdot (\text{\atppyrmal} \cdot \text{\amp}) +
3.31 \times 10^{-2} \cdot (\text{\atpsucc} \cdot \text{\mda}) \\
& 1.17 \cdot (\text{\popyrmal} \cdot \text{\atp}) -
1.45 \times 10^{-2} \cdot (\text{\popyrmal} \cdot \text{\ldh}) -\\
& 6.73 \times 10^{-1} \cdot (\text{\atp} \cdot \text{\amp}) +
5.03 \times 10^{-1} \cdot (\text{\atp} \cdot \text{\mda}) +\\
& 9.52 \times 10^{-2} \cdot (\text{\amp} \cdot \text{\ldh}) +
5.30 \times 10^{-1} \cdot (\text{\atpamp} \cdot \text{\mda}) -\\
& 2.28 \times 10^{-2} \cdot (\text{\atpamp} \cdot \text{\ldh}) +
8.08 \times 10^{-4} \cdot (\text{\mda} \cdot \text{\ldh}) +\\
& 1.84 \times 10^{-4} \cdot \text{\ldh}^2 +
29.1
\end{aligned}
\end{equation}



\section{Conclusions and future works} \label{sec:frassoni_conclusions}
% CCS
%future: collect more data (see learning curves with polyfeat) in order to have a uniformly distributed dataset
This chapter presented the first biomedical data challenge of the thesis. The goal of such task is to investigate on the changes of energy metabolism during the physiological aging by means of a set of metabolic biomarkers obtained from mononuclear cells isolated from peripheral blood.

After a preliminary, and insightful EDA, we realized that the measured variables are highly correlated and that they also show a non negligible trend with the age. Therefore, we investigated on the use of several linear and nonlinear regression models used in combination with a degree $2$ polynomial feature expansion.

We devised a Lasso-based regression model that, once trained on the $75\%$ of the dataset, predicted the age of the remaining $25\%$ with MAE of $6.591$ years ($\text{MSE}=82.182$, $\text{R}^2=0.901$ and $\text{EV}=0.902$). This result is in line with what observed in the experiments summarized in Table~\ref{tab:frassoni_model_challenge}.

\todo{...da qui in poi mi serve Silvia...}
To the best of our knowledge this is the first attempt to build an age predicting model that takes into account such metabolic biomarkers.
Our goal here is not to devise an accurate and ready-to-use age prediction model, whereas we aimed at verifying that the energetic state of blood cells by itself is already a good predictor of the age of a subject.

In the next future we plan to extend this study by enrolling more volunteers in order to expand the number of samples in the dataset.
We also plan to apply such method to a different cohort of adult patients that suffered from blood tumor in their childhood and that are currently in remission.



























%%
