% !TEX root = ../main.tex

\chapter{Model for metabolic age prediction} \label{chap:frassoni}
% Frassoni sani + CCS

%Per rispondere alla richiesta di Samuele:
%- MDA (malondialdeide) è un marcatore della perossidazione lipidica, quindi dello stress ossidativo
%- Consumo d'ossigeno e ATP sintesi con P/M: valutazione del consumo di ossigeno e sintesi di ATP in presenza di piruvato e malato che stimolano la via mitocondriale composta da Complesso I, III e IV (valuta la funzionalità del mitocondrio)
%- Consumo d'ossigeno e ATP sintesi con succinato: valutazione del consumo di ossigeno e sintesi di ATP in presenza di succinato che stimola la via mitocondriale composta da Complesso II, III e IV (valuta la funzionalità del mitocondrio)
%- ATP: concentrazione intracellulare di ATP
%- AMP: concentrazione intracellulare di AMP
%- ATP/AMP: rapporto tra le due precedenti  misure che indica lo stato energetico cellulare ( più elevato più la cellula possiede energia)
%- LDH: attività della lattico deidrogenasi, ultimo enzima del pathway della glicolisi anaerobia (antagonista della fosforilazione ossidativa che noi valutiamo tramite i consumo d'ossigeno e la sintesi di ATP).

\begin{displayquote}
	\textit{In this chapter, we evaluate the changes of energy metabolism during the physiological aging. To this aim we will develop a ML model that predicts the age of an individual starting from the following molecular biomarkers: oxidative phosphorylation efficiency, the \ac{ATP}/\ac{AMP} ratio, the lactate dehydrogenase activity and the level of malondialdehyde.  For this purpose we employed mononuclear cells obtained from healthy population with age between $5$ and $106$ years.}
\end{displayquote}

\section{Introduction: aging and metabolism} \label{sec:frassoni_intro}

In this chapter we present the first biomedical data science problem of the thesis. The problem consists in analyzing a set of molecular biomarkers collected from $118$ volunteers with the final aim of devising a ML model capable of predicting the age of an individual\footnote{this was referred to as \textit{the aging problem} throughout Chapter~\ref{chap:state-of-the-art}}. In order to achieve such model, extensive EDA  and thorough model selection will be performed.
Before diving into the experimental setup, let's see some preliminary biological notions of how aging influences our metabolism.

Aging is a multifactorial process characterized by a progressive decline of physiological functions~\cite{campisi2013aging}, which leads to an increment of vulnerability and the relative risk of disease and death ~\cite{bratic2010mitochondrial}.

Aging represents the primary risk factor for several chronic pathologies, \ie cancer, cardiovascular disorders, diabetes and neurodegeneration~\cite{lopez2013hallmarks}. Different molecular pathways seem involved in the aging process, including deregulated autophagy, mitochondrial dysfunction, telomere shortening, oxidative stress, systemic inflammation, and metabolism dysfunction~\cite{lopez2013hallmarks, riera2016signaling}.

Recently, an involvement of epigenetic modifications has been proposed~\cite{thompson2017epigenetic}, developing an \textit{aging clock} based on the degree of DNA methylation, which increases with the age~\cite{horvath2013dna}. However, for several years, aging has been considered the result of damages accumulation due to an excessive production of reactive oxygen species (ROS).

The \textit{Mitochondrial Theory of Aging}~\cite{harman1972biologic, sastre2000mitochondrial} derives from the  concept that mitochondria are the main source of oxidative stress~\cite{cadenas2000mitochondrial, turrens2003mitochondrial, dai2014mitochondrial} and the fact that mitochondrial DNA displays a
great rate of mutation together with a less efficient repair machinery with respect to nuclear DNA~\cite{short2005decline}. After some mitochondrial DNA mutation threshold, irreversible oxidative damages propagate throughout the genome. This phenomenon leads to dysfunction of mitochondrial metabolism~\cite{genova2004mitochondrial} accelerating the oxidative stress production~\cite{wallace2010mitochondrial}.

As shown in~\cite{mckerrell2015leukemia}, mononuclear cells isolated from peripheral blood, are an excellent model to evaluate the metabolic status of an entire organism. In fact, the molecular alterations identified in peripheral blood cells of aged normal subjects are statistically correlated with degenerative diseases~\cite{jaiswal2014age}.

%Moreover, we have trained on the collected data a machine learning model that aims at predicting the age of an individual based on its metabolic markers (features). Thanks to the adopted shrinkage penalties, the obtained model is at the same time robust to the noise in the data and capable of exploiting the collinearity among the features \cite{hastie2009elements}. The statistical soundness of the obtained model is thoroughly tested by extensive data resampling and refitting strategies \cite{molinaro2005prediction}. The achieved model attains an expected error of XXX years.

\section{Data collection} \label{sec:frassoni_data_collection}

The study presented in this chapter is performed on mononuclear cells isolated from peripheral blood obtained from a population of $118$ volunteers\footnote{all participants provided their written informed consent to participate in this study, which was approved by the Ethics Committee of the IRCCS Istituto G. Gaslini, Genoa, IT, \todo{(N°)}} with age between $5$ and $106$ years.
In order to preserve the collected blood samples, the vacutainer tubes were transferred into the laboratory and analyzed within $24$ hours from collection.
All chemicals were purchased from Sigma Aldrich (St. Louis, MO, USA), unless otherwise indicated. Ultrapure water (Milli-Q; Millipore, Billerica, MA, USA) was used throughout. All other reagents were of analytical grade.
Data collection and further analysis were managed by a team of specialized biologists at the IRCCS Istituto G. Gaslini, Genoa, IT. On each blood sample, we performed the following measures.

\begin{itemize}
	\item[] \textbf{ATP} This complex molecule is the main responsible for storing and exchanging energy in cells and it is often referred to as the \textit{energy currency} of the cell. From a chemical point of view, ATP is made of an adenine base attached to a ribose sugar, which, in turn, is attached to three phosphate groups.
	ATP is heavily involved in different cellular aerobic respiration pathways. High levels of ATP correspond to high energetic state.
	ATP intracellular concentration, measured in $\text{mM}/\text{m}$, is an important molecular biomarker to evaluate the energetic state of a cell.
	
	\item[] \textbf{AMP} This molecule is one of the main derivatives of ATP. In fact, AMP ca be obtained when two phosphate groups are removed from ATP, releasing energy that can be transferred to other molecules to trigger some further cell reaction.
	Therefore, when a cell is in good health, \ie high energetic level, AMP is low (and ATP is high).
	AMP cellular concentration is measured in $\text{mM}/\text{m}$ and it is considered as an important molecular biomarker for the cellular energetic state. ATP and AMP quantification was based on the enzyme coupling method presented in~\cite{ravera2013tricarboxylic}.
%	as a side product of the ATP synthesis process or 
	
	\item[] \textbf{ATP/AMP ratio} With the aim of predicting the age of an individual by assessing the energetic state of the peripheral blood cells, measuring ATP and AMP cellular concentration may not be enough. A more representative and interesting quantity can be their ratio, so ATP/AMP ratio was calculated and added to the feature set.
	
	\item[] \textbf{Oxygen consumption} Aerobic cellular respiration requires oxygen to produce ATP. Therefore, the cellular oximetric level is an important molecular biomarker for the cellular metabolic assessment. Oxygen consumption was measured with an amperometric electrode in a closed chamber, magnetically stirred, at 37°C. The oxygen consumption measure, expressed in $\text{nmol~O}_2/(\text{min}\cdot\text{mg})$, is repeated in two versions, adding two different substrates, \ie
	\begin{enumerate*}[label=(\roman*)]
		\item a combination of $5~\text{mM}$ pyruvate with $2.5~\text{mM}$ malate or
		\item only $20~\text{mM}$ succinate.
	\end{enumerate*}
	In the first oximetric measure, which from now on will be referred to as \copyrmal, the substrate stimulates the pathway composed by Complexes I, III and IV. On the other hand, in the second oximetric measure, which we call \cosucc, the substrate activates the pathway composed by Complexes II, III and IV, as described in~\cite{cappelli2017defects}.
	
	\item[] \textbf{ATP synthesis} We already covered the importance of the ATP for the evaluation of the metabolic state of cells. We also measured ATP synthesis, expressed in $\text{nmol~ATP}/(\text{min}\cdot\text{mg})$,  by the highly sensitive luciferin/luciferase method. The same two substrates used for the oximetric evaluation were adopted. In the remainder of the chapter we refer to such measures as \atppyrmal and \atpsucc, accordingly.
	
	\item[] \textbf{P/O ratio} In order to assess the efficiency of oxidative phosphorylation, we also evaluated the ratio between ATP synthesis and oxygen consumption under both the substrates. We call the two obtained features as \popyrmal and \posucc, respectively.
	
	\item[] \textbf{Glycolytic flux} In order to assay the glycolytic flux, we measured the activity of the Lactate Dehydrogenase (\ac{LDH}), expressed in $\text{mU}/\text{mg}$. This enzyme is important to evaluate the other metabolic pathway not mentioned so far, \ie the anaerobic respiration.
	
	\item[] \textbf{Lipid peroxidation} The uncoupled oxidative phosphorylation metabolism is often associated with an increment in the oxidative stress production~\cite{dai2014mitochondrial}, which induces damages on proteins, nucleic acid and membrane. Therefore, we evaluated the level of malondialdehyde (\ac{MDA}) as a marker of lipid peroxidation. The measure of \mda follows the protocol in~\cite{ravera2015oxidative}.	
\end{itemize}

Each sample of this dataset is then described by the $12$-dimensional feature set summarized in Table~\ref{tab:aging_features}.

\begin{table}[]
	\centering
	\caption{The $12$-dimensional feature set.}
	\label{tab:aging_features}
	\begin{tabular}{@{}ll@{}}
		\toprule
		\textbf{Measure}                                    & \textbf{Feature name}\\ \midrule
		Gender of the individual                      & sex          \\
		ATP intracellular concentration            & \atp         \\
		AMP intracellular concentration            & \amp         \\
		ATP/AMP ratio                              & \atpamp      \\
		Oxygen consumption under Pyruvate + Malate & \copyrmal    \\
		Oxygen consumption under Succinate         & \cosucc      \\
		ATP synthesis under Pyruvate + Malate      & \atppyrmal   \\
		ATP synthesis under Succinate              & \atpsucc     \\
		P/O ratio under Pyruvate + Malate          & \popyrmal    \\
		P/O ratio under Succinate                  & \posucc      \\
		Glycolytic flux                            & \ldh         \\
		Lipid peroxidation                         & \mda         \\ \bottomrule
	\end{tabular}
\end{table}


\section{Exploratory data analysis} \label{sec:frassoni_EDA}
% BOXPLOT - PCA - HEATMAP etc

\section{Metabolic age prediction} \label{sec:frassoni_regression}
% pipeline + assessment + model challenge

\section{Results} \label{sec:frassoni_results}
% tabella dei risultati

\section{Conclusions and future works} \label{sec:frassoni_conclusions}
% CCS


%%