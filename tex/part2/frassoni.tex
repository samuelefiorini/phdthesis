% !TEX root = ../main.tex

\chapter{Model for metabolic age prediction} \label{chap:frassoni}
% Frassoni sani + CCS

\begin{displayquote}
	\textit{abstract here.}
\end{displayquote}

\section{Aging and metabolism} \label{sec:frassoni_intro}

Aging is a multifactorial process characterized by a progressive decline of physiological functions~\cite{campisi2013aging}, which leads to an increment of vulnerability and the relative risk of disease and death ~\cite{bratic2010mitochondrial}.

Aging represents the primary risk factor for several chronic pathologies, \ie cancer, cardiovascular disorders, diabetes and neurodegeneration~\cite{lopez2013hallmarks}. Different molecular pathways seem involved in the aging process, including deregulated autophagy, mitochondrial dysfunction, telomere shortening, oxidative stress, systemic inflammation, and metabolism dysfunction~\cite{lopez2013hallmarks, riera2016signaling}.

Recently, an involvement of epigenetic modifications has been proposed~\cite{thompson2017epigenetic}, developing an \textit{aging clock} based on the degree of DNA methylation, which increases with the age~\cite{horvath2013dna}. However, for several years, aging has been considered the result of damages accumulation due to an excessive production of reactive oxygen species (ROS). The \textit{Mitochondrial Theory of Aging}~\cite{harman1972biologic, sastre2000mitochondrial} derives from the  concept that mitochondria are the main source of oxidative stress~\cite{cadenas2000mitochondrial, turrens2003mitochondrial, dai2014mitochondrial} and the fact that mitochondrial DNA displays a
great rate of mutation together with a less efficient repair machinery with respect to nuclear DNA~\cite{short2005decline}. After some mitochondrial DNA mutation threshold, irreversible oxidative damages propagate throughout the genome. This phenomenon leads to dysfunction of mitochondrial metabolism~\cite{genova2004mitochondrial} accelerating the oxidative stress production~\cite{wallace2010mitochondrial}.

In this chapter, we evaluate the changes of energy metabolism during the physiological aging by analyzing the oxidative phosphorylation efficiency, the ATP/AMP ratio, the lactate dehydrogenase activity and the MDA content.  For this purpose we employed mononuclear cells (MNC) isolated from peripheral blood (PB), obtained from healthy population with age between $5$ and $106$ years.

Moreover, we have trained on the collected data a machine learning model that aims at predicting the age of an individual based on its metabolic markers (features). Thanks to the adopted shrinkage penalties, the obtained model is at the same time robust to the noise in the data and capable of exploiting the collinearity among the features \cite{hastie2009elements}. The statistical soundness of the obtained model is thoroughly tested by extensive data resampling and refitting strategies \cite{molinaro2005prediction}. The achieved model attains an expected error of XXX years.
