% !TEX root = ./main.tex

\chapter{Introduction} \label{chapter:introduction}
% \addcontentsline{toc}{chapter}{\nameref{chapter:introduction}}

% cercare soluzione tramite metodi statistici di problemi complessi biomedicali con particolare attenzione ai time-evolving data
% qui no soluzioni ma problemi: domande + esempi
% struttura tesi

This PhD thesis is divided in two parts. Part~\ref{part:I} presents a thorough description of the multi-disciplinary prerequisites which are relevant for the comprehension of Part~\ref{part:II}, where the original contributions are described.

Part~\ref{part:I} is organized as follows: Chapter~\ref{chap:background} introduces the concept of \textit{data science} and its declination toward life science studies. It also describes the main challenges biomedical data scientists typically deal with along with few examples of the most common biological questions and how are they translated into data analysis tasks. Moreover, in Chapter~\ref{chap:state-of-the-art} an overview on the state of the art on the most common statistical and technological tools is presented.
